%%% PACKAGES %%%
\usepackage{amsmath}
\usepackage{amsthm}
\usepackage{amssymb}
\usepackage{amsfonts}
\usepackage{color}
\usepackage{microtype}
\usepackage{lmodern}
\usepackage[mathscr]{eucal}
\usepackage{bm}
\usepackage{enumitem}
\usepackage{graphicx}
% \usepackage[portuguese]{babel}
\usepackage{hyperref}
\usepackage{titlesec}
%\usepackage{makecell}
%\usepackage{colortbl}
%\usepackage[dvipsnames]{xcolor}



%%% MACROS %%%
% Math Black Board
\def\R{\mathbb{R}}
\def\Z{\mathbb{Z}}
\def\N{\mathbb{N}}
\def\Q{\mathbb{Q}}
\def\P{\mathbb{P}}
\def\H{\mathbb{H}}
\def\E{\mathbb{E} \hspace{0.2mm}}

% Math Caligraphic
\def\V{\mathcal{V}}
\def\B{\mathcal{B}}
\def\F{\mathcal{F}}

% Math Script
\def\U{\mathscr{U}}
\def\C{\mathscr{C}}
\def\X{\mathscr{X}}

\def\Rn{\R^n}
\def\inv{^{-1}}
\def\spam{\operatorname{spam}}
\def\ans{\textnormal{\textbf{Solução: }}}
\def\Id{\operatorname{Id}}
\def\limn{\lim_{n\rightarrow \infty}}
\def\and{\quad\text{and}\quad}
\def\opt{{\operatorname{opt}}}
\def\sign{{\operatorname{sign}}}

\newcommand{\ntexttt}[1]{\texttt{\textup{#1}}}
\newcommand{\abs}[1]{\left\vert#1\right\vert}
\newcommand{\inner}[1]{\left\langle #1\right\rangle}
\newcommand{\re}[1]{\operatorname{Re}(#1)}
\newcommand{\im}[1]{\operatorname{Im}(#1)}
\newcommand{\sgn}[1]{\operatorname{sgn}(#1)}
\newcommand{\norm}[1]{\left\lVert#1\right\rVert}
\newcommand{\normp}[2][p]{\parent{\int |#2|^#1}^{1/#1}}
\newcommand{\del}[2]{\dfrac{\partial #1}{\partial #2}}
\newcommand{\deldel}[3]{\dfrac{\partial^2 #1}{\partial #3\partial #2}}
\newcommand{\indicadora}[1]{\mathbf{1}_{(#1)}}
%\newcommand{\E}[1]{\mathbb{E}\left[#1\right]}
\newcommand{\var}[1]{\mathbb{V}\left[#1\right]}
\newcommand{\cov}[1]{\text{Cov}\left[#1\right]}
%\newcommand{\EP}[2]{\underset{#1}{\mathbb{E}}\left[#2\right]}
\newcommand{\EP}[2]{\mathbb{E}_{#1}\left[#2\right]}
\newcommand{\seq}[1]{(#1_n)_{n\in\N}}
\newcommand{\under}[2]{\underset{#2}{{\underbrace{#1}}}} % para escrever texto embaixo de um brace.
\DeclareMathOperator*{\argmax}{arg\,max}
\DeclareMathOperator*{\argmin}{arg\,min}



%%% SETTINGS %%%
% \setlist[enumerate,1]{label=\textnormal{(\alph*)}}
% \setlist[enumerate,2]{label=\textnormal{(\roman*)}}
% \renewcommand{\thesubsection}{\arabic{subsection}}
% %\renewcommand{\thesection}{}
% \renewcommand{\contentsname}{Contents}
% \renewcommand{\solutiontitle}{\noindent\textnormal{\nopagebreak[4]\textbf{Solution:}}\par\noindent}
% \renewcommand\partlabel{\textnormal{(\alph{partno})}}
% \graphicspath{ {./images/} }
% \usepackage[margin=1.1in]{geometry}
% \pagestyle{plain}
% \renewcommand\thesubsection{\thesection.\arabic{subsection}}
% \renewcommand\thesubsubsection{\thesubsection.\arabic{subsubsection}}


%%% COMMANDS %%%
% To add a section to the table of contents without numbering it
\newcommand*\silentsection[1]{%
\addtocounter{section}{1}
\addcontentsline{toc}{section}{\protect\numberline{\thesection}#1}
\sectionmark{#1}
}

% To add a subsection to the table of contents without numbering it
\newcommand*\silentsubsection[1]{%
  \addtocounter{subsection}{1}
  \addcontentsline{toc}{subsection}{\protect\numberline{\thesubsection}#1}
  \subsectionmark{#1}
}

% To add a subsubsection to the table of contents without numbering it
\newcommand*\silentsubsubsection[1]{%
  \addtocounter{subsubsection}{1}
  \addcontentsline{toc}{subsubsection}{\protect\numberline{\thesubsubsection}#1}
  \subsubsectionmark{#1}
}

%%% LISTINGS %%%
\usepackage{listings}
\lstset{ 
  language=Python,
  basicstyle=\normalfont\ttfamily\small,
  numbers=none,
  frame=lines,
  framexleftmargin=0.5em,
  xleftmargin=0.5em,
  framexrightmargin=0.5em,
  rulecolor=\color{black},
  belowcaptionskip=1\baselineskip,
  breaklines=true,
  showstringspaces=false,
  belowskip=1em,
  aboveskip=1em,
  linewidth=\linewidth,
  float=htb
  captionpos=b
  morekeywords={as}, % Add 'as' as an additional keyword
  keywordstyle=\color{blue}\textbf, % Set the color for keywords
   literate={á}{{\'a}}1
           {â}{{\^a}}1
           {ã}{{\~a}}1
           {à}{{\`a}}1
           {é}{{\'e}}1
           {ê}{{\^e}}1
           {í}{{\'i}}1
           {ó}{{\'o}}1
           {ô}{{\^o}}1
           {õ}{{\~o}}1
           {ú}{{\'u}}1
           {ü}{{\"u}}1
           {ç}{{\c{c}}}1
}



% % Questions in italic
% \let\oldquestions\questions
% \let\endoldquestions\endquestions
% \renewenvironment{questions}
%   {\oldquestions\itshape}
%   {\endoldquestions}
  
% % Solutions not in italic
% \let\oldsolution\solution
% \let\endoldsolution\endsolution
% \renewenvironment{solution}{%
%   \begin{oldsolution}%
%   \normalshape% Make the text inside italic
% }{%
%   \end{oldsolution}%
% }
